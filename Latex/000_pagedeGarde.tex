    %%%%%%%%
    %Cette partie sert à la création de la page de garde
    %%%%%%%%
    
    %%%%%%%%
    %Hypersetup permet de définir les couleurs des liens hypertexts
    %%%%%%%%
    \hypersetup{
        colorlinks=true,          % false: liens encadrés; true: liens colorés
        linkcolor=blue,          % couleur des liens (ou bordures) internes
        citecolor=red,          % couleur des liens (ou bordures) vers bilbio
        filecolor=black,          % couleur des liens (ou bordures) vers fichiers
        urlcolor=black            % couleur des liens (ou bordures) url
    }
    
    %%%%%%%%
    %Géométrie permet de modifier la mise en page 
    %%%%%%%%
     %\geometry{
     %a4paper,
     %total={170mm,257mm},
     %left=20mm,
     %top=20mm,
     %}
    
    
    \parindent=0em
    \pagestyle{fancy}
    
    \renewcommand{\headrulewidth}{0pt}
    
    %%%%%%%%%
    %Pour avoir des header droite gauche sur les pages du documents
    %%%%%%%%%
    \fancyhead[L]{}   %Header de gauche
    \fancyhead[R]{MSc Project 2017-2018 CHYN}   %Header de droite
    
    %%%%%%%%%
    %Pour la première page seulement
    %%%%%%%%%
    \fancypagestyle{firstpage}{
     \rhead{\vspace{20 mm}\includegraphics[width = 70mm]{logo/Dauphine_new.png}} %Logo de l'uni a droite
      \lhead{\textit{Centre d'hydrogéologie et de géothermie                     %Texte de gauche 
    	\\ Université de Neuchâtel}}\fancyfoot{}}                                % \\ = saut de ligne
    
    %%%%%%%%%
    %Titre
    %Vspace pour mettre de l'écart entre les lignes
    %%%%%%%%%
    \title{\vspace{50 mm}\rule{\linewidth}{.5pt} \textit{Master en Hydrogéologie et Géothermie}\vspace{\fill}\\\rule{\linewidth}{.5pt}\\\textbf{
    Template Projet Master}
    \\\rule{\linewidth}{.5pt}\\ \vspace{20 mm}}
    
    % \small ou \Large pour modifier localement la taille du texte
    
    \author{ \Large{Valentin Dall'Alba}\\\small{Supervised by Prof. ---- } \\\small{and-----} \vspace{\fill}}
    
    
    
    
    
    
