\subsection{Estimation des modèles de probabilité linéaire, du logit et du probit.}

\subsubsection{Écriture des différentes modèles.}

\subsubsection*{Modèle de probabilité linéaire :} 

\begin{align}
    y_i &= X \beta + \varepsilon_i \nonumber
\end{align}

\begin{align}
    y_i &=  \beta_0 + \beta_1 \cdot nwifeinc_i + \beta_2 \cdot educ_i + \beta_3 \nonumber \cdot exper_i + \beta_4 \cdot exper_i^2 \\ 
    & + \beta_5 \cdot age_i + \beta_6 \cdot kidslt6_i + \beta_7 \cdot kidsge6_i + \varepsilon_i \nonumber
\end{align}

\subsubsection*{Modèle Logit :} 

\begin{align}
    p_i = P(Y_i = 1) &= P(y_i^* | X_i = x_i > 0) \nonumber  \\ 
               &=  P(X \beta + \varepsilon_i | X_i = x_i > 0 ) \nonumber  \\ 
               &= P(\varepsilon_i | X_i = x_i > - X \beta) \nonumber && \\ 
               &= 1 - F(- X_i \beta) \nonumber && \\
               &= F(X_i \beta) \nonumber && \\ 
               &=  \frac{e^{X_i \cdot \beta}}{1 + e^{X_i \cdot \beta}}  
\end{align}

Nous obtenons : 

\begin{align}
    p_i = \frac{e^{\beta_0 + \beta_1 \cdot nwifeinc_i + \beta_2 \cdot educ_i + \beta_3 \cdot exper_i + \beta_4 \cdot exper_i^2 + \beta_5 \cdot age_i + \beta_6 \cdot kidslt6_i + \beta_7 \cdot kidsge6_i}}{1 + e^{\beta_0 + \beta_1 \cdot nwifeinc_i + \beta_2 \cdot educ_i + \beta_3 \cdot exper_i + \beta_4 \cdot exper_i^2 + \beta_5 \cdot age_i + \beta_6 \cdot kidslt6_i + \beta_7 \cdot kidsge6_i}}
\end{align}

\subsubsection*{Modèle Probit :} 

Dans le cadre du Probit, nous utilisons la fonction de répartition suivante : 

\[ p_i = \int_{-\infty}^{\infty} \frac{1}{\sqrt{2 \pi}} e^{\frac{x^2}{2}} \,dx \]


% \[ p_i = \int_{-\infty}^{\infty} \frac{1}{\sqrt{2 \pi}} e^{\frac{(\beta_0 + \beta_1 \cdot nwifeinc_i + \beta_2 \cdot educ_i + \beta_3 \cdot exper_i + \beta_4 \cdot exper_i^2 + \beta_5 \cdot age_i + \beta_6 \cdot kidslt6_i + \beta_7 \cdot kidsge6_i)^2}{2}} \,dx \]

\subsubsection{Les estimations des trois modèles.}






